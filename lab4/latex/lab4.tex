\documentclass{extreport}

\usepackage[14pt]{extsizes}
\usepackage[T2A]{fontenc}
\usepackage[utf8]{inputenc}
\usepackage[english,ukrainian]{babel}

\usepackage[a4paper, top=10mm, bottom=15mm, left=20mm, right=15mm]{geometry}
\usepackage{amsmath,amsfonts,amssymb,amsthm,mathtools}
\usepackage{listings}
\usepackage{graphicx}
\usepackage{enumitem}
\usepackage{verbatim}
\usepackage{listings}
\usepackage{xcolor}
\usepackage{textgreek}
\usepackage{diagbox}
\usepackage{pgfplots}
\pgfplotsset{compat = 1.16}


\lstdefinestyle{output_txt}{
    basicstyle=\ttfamily\footnotesize,
    breakatwhitespace=false,         
    breaklines=true,                 
    captionpos=b,                    
    keepspaces=true,                                    
    numbersep=5pt,                  
    showspaces=false,                
    showstringspaces=false,
    showtabs=false,                  
    tabsize=2
}
\definecolor{codegreen}{rgb}{0,0.6,0}
\definecolor{codegray}{rgb}{0.5,0.5,0.5}
\definecolor{codepurple}{rgb}{0.58,0,0.82}
\lstdefinestyle{python_code}{ 
    commentstyle=\color{codegreen},
    keywordstyle=\color{magenta},
    numberstyle=\tiny\color{codegray},
    stringstyle=\color{codepurple},
    basicstyle=\ttfamily\footnotesize,
    breakatwhitespace=false,         
    breaklines=true,                 
    captionpos=b,                    
    keepspaces=true,                            
    numbersep=5pt,                  
    showspaces=false,                
    showstringspaces=false,
    showtabs=false,                  
    tabsize=4
}

\setlist[enumerate]{nosep}
\graphicspath{{pics/}}
\DeclareGraphicsExtensions{.png}

\begin{document}
\begin{titlepage}
    \thispagestyle{empty}
    \begin{center}
        \includegraphics[width = \textwidth]{kpi}
        Міністерство освіти і науки України\\
        Національний технічний університет України\\
        <<Київський політехнічний інститут ім. І. Сікорського>>\\
        Інститут прикладного системного аналізу
    \end{center}
    \vspace{40mm}
    \begin{center}
        \textbf{Лабораторна робота} \\
        з курсу <<Методи оптимізації>> \\
        з теми <<Методи одновимірної оптимізації>
    \end{center}
    \vspace{20mm}
    \begin{flushleft}
        Виконали студенти 3 курсу групи КА-81 \\
        Галганов Олексій \\
        Єрко Андрій \\
        Фордуй Нікіта
    \end{flushleft}
    \begin{flushright}
        Перевірили \\
        Спекторський Ігор Якович \\
        Яковлева Алла Петрівна
    \end{flushright}
    \vspace{30mm}
    \begin{center}
        \textbf{Київ 2021}
    \end{center}
\end{titlepage}

%\begin{center}
%    \textbf{Варіант 1}
%\end{center}
\noindent\textbf{Завдання.} Порівняти різні методи одновимірної оптимізації.

\noindent\textbf{Результати роботи.}
Ми вирішили порівняти методи дихотомії, золотого перетину та Фібоначчі, а також методи, що можуть застосовуватися до функцій довільної
кількості змінних --- градієнтний метод та метод Ньютона з дробленням кроку.
Для методу Фібоначчі в таблицях як кількість ітерацій пишемо мінімальну кількість ітерацій, яка дала розв'язок, наближено рівний іншим.
\begin{center}
    $f(x) = x^2 + x$

    \begin{tikzpicture}
        \begin{axis}[axis lines = center]
        \addplot [
            domain=-2:2, 
            samples=100, 
            color=black,
        ] {x^2 + x};
        \end{axis}
    \end{tikzpicture}

    \begin{tabular}{|c|c|c|c|}
    \hline
                    \bf метод & $x_0$ або $[a_0; b_0]$ &  \bf кінцева точка &  \bf к-сть ітерацій \\
                    \hline
              градієнтний &                    $2$ &    $-0.500000$ &                   1 \\
              \hline
                  Ньютона &                    $2$ &    $-0.499995$ &                  19 \\
                  \hline
                дихотомії &              $[-2; 2]$ &    $-0.500003$ &                  18 \\
                \hline
        золотого перетину &              $[-2; 2]$ &    $-0.499998$ &                  26 \\
        \hline
                Фібоначчі &              $[-2, 2]$ &    $-0.500004$ &                  15 \\
                \hline
    \end{tabular}
\end{center}

\begin{center}
    $f(x) = (x-1)^6$

    \begin{tikzpicture}
        \begin{axis}[axis lines = center]
        \addplot [
            domain=0:2, 
            samples=100, 
            color=black,
        ] {(x-1)^6};
        \end{axis}
    \end{tikzpicture}
    
    \begin{tabular}{|c|c|c|c|}
    \hline
                    \bf метод & $x_0$ або $[a_0; b_0]$ &  \bf кінцева точка &  \bf к-сть ітерацій \\
                    \hline
              градієнтний &                    $0$ &    $1.095285$ &                   1000 \\
              \hline
                  Ньютона &                    $0$ &    $0.935815$ &                  26 \\
                  \hline
                дихотомії &              $[0; 2]$ &    $0.999993$ &                  17 \\
                \hline
        золотого перетину &              $[0; 2]$ &    $0.999999$ &                  24 \\
        \hline
                Фібоначчі &              $[0, 2]$ &    $1.001520$ &                  15 \\
                \hline
    \end{tabular}
\end{center}

\begin{center}
    $f(x) = \sqrt{|x+3|} - \cos^4 x$
    \nopagebreak

    \begin{tikzpicture}
        \begin{axis}[axis lines = center]
        \addplot [
            domain=-4:-2, 
            samples=200, 
            color=black,
        ] {sqrt(abs(x+3)) - (cos(deg(x)))^4};
        \end{axis}
    \end{tikzpicture}

    \vspace{0.5em}

    \begin{tabular}{|c|c|c|c|}
    \hline
                    \bf метод & $x_0$ або $[a_0; b_0]$ &  \bf кінцева точка &  \bf к-сть ітерацій \\
                    \hline
              градієнтний &                    $-4$ &    $-3.000058$ &                   1000 \\
              \hline
                  Ньютона &                    $-4$ &    $-4.000032$ &                  1000 \\
                  \hline
                дихотомії &              $[-4; -2]$ &    $-2.999993$ &                  17 \\
                \hline
        золотого перетину &              $[-4; -2]$ &    $-3.0$ &                  24 \\
        \hline
                Фібоначчі &              $[-4; -2]$ &    $-3.000505$ &                  15 \\
                \hline
    \end{tabular}
\end{center}

\begin{center}
    $f(x) = e^x - 0.33x^3 + 2x$

    \begin{tikzpicture}
        \begin{axis}[axis lines = center]
        \addplot [
            domain=-3:0, 
            samples=100, 
            color=black,
        ] {exp(x) - 0.33*x^3 + 2*x};
        \end{axis}
    \end{tikzpicture}
    
    \vspace{0.5em}

    \begin{tabular}{|c|c|c|c|}
    \hline
                    \bf метод & $x_0$ або $[a_0; b_0]$ &  \bf кінцева точка &  \bf к-сть ітерацій \\
                    \hline
              градієнтний &                    $0$ &    $-1.498630$ &                   12 \\
              \hline
                  Ньютона &                    $0$ &    $-1.498631$ &                  10 \\
                  \hline
                дихотомії &              $[-3; 0]$ &    $-1.498627$ &                  18 \\
                \hline
        золотого перетину &              $[-3; 0]$ &    $-1.498628$ &                  25 \\
        \hline
                Фібоначчі &              $[-3; 0]$ &    $-1.498622$ &                  25 \\
                \hline
    \end{tabular}
\end{center}

\noindent\textbf{Лістинг.}
%Текст програми було розділено на \texttt{Optimizer.py} з реалізацією
%власне методу проекції градієнту і методу Ньютона з попередньої лабораторної роботи, який використовується
%для розв'язання задачі проекції, та \texttt{lab3.py}, де викликаються необхідні функції та зберігаються результати.

%\noindent\texttt{Optimizer.py}
%\lstinputlisting[language = Python, style = python_code]{../code/Optimizer.py}

%\noindent\texttt{lab4.py}
%\lstinputlisting[language = Python, style = python_code]{../code/lab3.py}

\noindent\textbf{Висновки.}
\end{document}