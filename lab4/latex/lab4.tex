\documentclass{extreport}

\usepackage[14pt]{extsizes}
\usepackage[T2A]{fontenc}
\usepackage[utf8]{inputenc}
\usepackage[english,ukrainian]{babel}

\usepackage[a4paper, top=10mm, bottom=15mm, left=20mm, right=15mm]{geometry}
\usepackage{amsmath,amsfonts,amssymb,amsthm,mathtools}
\usepackage{listings}
\usepackage{graphicx}
\usepackage{enumitem}
\usepackage{verbatim}
\usepackage{listings}
\usepackage{xcolor}
\usepackage{textgreek}
\usepackage{diagbox}
\usepackage{pgfplots}
\pgfplotsset{compat = 1.16}


\lstdefinestyle{output_txt}{
    basicstyle=\ttfamily\footnotesize,
    breakatwhitespace=false,         
    breaklines=true,                 
    captionpos=b,                    
    keepspaces=true,                                    
    numbersep=5pt,                  
    showspaces=false,                
    showstringspaces=false,
    showtabs=false,                  
    tabsize=2
}
\definecolor{codegreen}{rgb}{0,0.6,0}
\definecolor{codegray}{rgb}{0.5,0.5,0.5}
\definecolor{codepurple}{rgb}{0.58,0,0.82}
\lstdefinestyle{python_code}{ 
    commentstyle=\color{codegreen},
    keywordstyle=\color{magenta},
    numberstyle=\tiny\color{codegray},
    stringstyle=\color{codepurple},
    basicstyle=\ttfamily\footnotesize,
    breakatwhitespace=false,         
    breaklines=true,                 
    captionpos=b,                    
    keepspaces=true,                            
    numbersep=5pt,                  
    showspaces=false,                
    showstringspaces=false,
    showtabs=false,                  
    tabsize=4
}

\setlist[enumerate]{nosep}
\graphicspath{{pics/}}
\DeclareGraphicsExtensions{.png}

\begin{document}
\begin{titlepage}
    \thispagestyle{empty}
    \begin{center}
        \includegraphics[width = \textwidth]{kpi}
        Міністерство освіти і науки України\\
        Національний технічний університет України\\
        <<Київський політехнічний інститут ім. І. Сікорського>>\\
        Інститут прикладного системного аналізу
    \end{center}
    \vspace{40mm}
    \begin{center}
        \textbf{Лабораторна робота} \\
        з курсу <<Методи оптимізації>> \\
        з теми <<Методи спряжених градієнтів>
    \end{center}
    \vspace{20mm}
    \begin{flushleft}
        Виконали студенти 3 курсу групи КА-81 \\
        Галганов Олексій \\
        Єрко Андрій \\
        Фордуй Нікіта
    \end{flushleft}
    \begin{flushright}
        Перевірили \\
        Спекторський Ігор Якович \\
        Яковлева Алла Петрівна
    \end{flushright}
    \vspace{30mm}
    \begin{center}
        \textbf{Київ 2021}
    \end{center}
\end{titlepage}

\begin{center}
    \textbf{Варіант 1}
\end{center}
\noindent\textbf{Завдання.} Скласти програму для мінімізації цільової функції методом спряжених градієнтів.

\emph{Цільова функція:}
$f(x,y) = x^2 + 18y^2 + 0.01xy + x - y$

\noindent\textbf{Результати роботи.}
Цільова функція є квадратичною: $$f(x, y) = \frac{1}{2} \left<\begin{pmatrix}
    2 & 0.01 \\ 0.01 & 36
\end{pmatrix} \begin{pmatrix}
    x \\ y
\end{pmatrix}, \begin{pmatrix}
    x \\y
\end{pmatrix} \right> + 
\left<\begin{pmatrix}
    1 \\ -1
\end{pmatrix}, \begin{pmatrix}
    x \\y
\end{pmatrix} \right>$$
Мінімум --- у точці $\begin{pmatrix}
    -0.50014 \\
    0.0279167
\end{pmatrix}$.
Метод спряжених напрямків має мінімізувати її за не більше ніж 2 кроки. Дійсно, при запуску методу з випадкових початкових точок
отримали точку мінімуму за два кроки.
\begin{center}
    \begin{tabular}{c c}
        \includegraphics[scale = 0.5]{quadratic_random_1.png} &
        \includegraphics[scale = 0.5]{quadratic_random_2.png} \\
        \includegraphics[scale = 0.5]{quadratic_random_3.png} &
        \includegraphics[scale = 0.5]{quadratic_random_4.png} \\
        \includegraphics[scale = 0.5]{quadratic_random_5.png} &
        \includegraphics[scale = 0.5]{quadratic_random_6.png} \\
    \end{tabular}
\end{center}

Ми також вирішили перевірити цей метод на функції Розенброка
$f(x, y) = (y-x^2)^2 + (1-x)^2$, яка вже не є квадратичною. Вона має мінімум у точці $(1, 1)$.
Перевірку зробили з чотирьох початкових точок.
\begin{center}
    \begin{tabular}{c c}
        \includegraphics[scale = 0.5]{rosenbrok_0.png} &
        \includegraphics[scale = 0.5]{rosenbrok_1.png} \\
        \includegraphics[scale = 0.5]{rosenbrok_10.png} &
        \includegraphics[scale = 0.5]{rosenbrok_100.png} \\
    \end{tabular}
\end{center}

\noindent\textbf{Лістинг.}
%Текст програми було розділено на \texttt{Optimizer.py} з реалізацією
%власне методу проекції градієнту і методу Ньютона з попередньої лабораторної роботи, який використовується
%для розв'язання задачі проекції, та \texttt{lab3.py}, де викликаються необхідні функції та зберігаються результати.

%\noindent\texttt{Optimizer.py}
%\lstinputlisting[language = Python, style = python_code]{../code/Optimizer.py}

%\noindent\texttt{lab4.py}
%\lstinputlisting[language = Python, style = python_code]{../code/lab3.py}

\noindent\textbf{Висновки.}
\end{document}