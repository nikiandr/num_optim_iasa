\documentclass{extreport}

\usepackage[14pt]{extsizes}
\usepackage[T2A]{fontenc}
\usepackage[utf8]{inputenc}
\usepackage[english,ukrainian]{babel}

\usepackage[a4paper, top=10mm, bottom=15mm, left=20mm, right=15mm]{geometry}
\usepackage{amsmath,amsfonts,amssymb,amsthm,mathtools}
\usepackage{listings}
\usepackage{graphicx}
\usepackage{enumitem}
\usepackage{verbatim}
\usepackage{listings}
\usepackage{xcolor}
\usepackage{textgreek}
\usepackage{diagbox}
\usepackage{pgfplots}

\pgfplotsset{compat = 1.16}


\lstdefinestyle{output_txt}{
    basicstyle=\ttfamily\footnotesize,
    breakatwhitespace=false,         
    breaklines=true,                 
    captionpos=b,                    
    keepspaces=true,                                    
    numbersep=5pt,                  
    showspaces=false,                
    showstringspaces=false,
    showtabs=false,                  
    tabsize=2
}
\definecolor{codegreen}{rgb}{0,0.6,0}
\definecolor{codegray}{rgb}{0.5,0.5,0.5}
\definecolor{codepurple}{rgb}{0.58,0,0.82}
\lstdefinestyle{python_code}{ 
    commentstyle=\color{codegreen},
    keywordstyle=\color{magenta},
    numberstyle=\tiny\color{codegray},
    stringstyle=\color{codepurple},
    basicstyle=\ttfamily\footnotesize,
    breakatwhitespace=false,         
    breaklines=true,                 
    captionpos=b,                    
    keepspaces=true,                            
    numbersep=5pt,                  
    showspaces=false,                
    showstringspaces=false,
    showtabs=false,                  
    tabsize=4
}

\setlist[enumerate]{nosep}
\graphicspath{{pics/}}
\DeclareGraphicsExtensions{.png}

\begin{document}
\begin{titlepage}
    \thispagestyle{empty}
    \begin{center}
        \includegraphics[width = \textwidth]{kpi}
        Міністерство освіти і науки України\\
        Національний технічний університет України\\
        <<Київський політехнічний інститут ім. І. Сікорського>>\\
        Інститут прикладного системного аналізу
    \end{center}
    \vspace{40mm}
    \begin{center}
        \textbf{Лабораторна робота} \\
        з курсу <<Методи оптимізації>> \\
        з теми <<Метод лінеаризації в нелінійному програмуванні>>
    \end{center}
    \vspace{20mm}
    \begin{flushleft}
        Виконали студенти 3 курсу групи КА-81 \\
        Галганов Олексій \\
        Єрко Андрій \\
        Фордуй Нікіта
    \end{flushleft}
    \begin{flushright}
        Перевірили \\
        Спекторський Ігор Якович \\
        Яковлева Алла Петрівна
    \end{flushright}
    \vspace{30mm}
    \begin{center}
        \textbf{Київ 2021}
    \end{center}
\end{titlepage}


\noindent\textbf{Завдання.} Реалізувати метод лінеаризації для задачі мінімізації нелінійної функції на множині, що задана набором нелінійних нерівностей:
\begin{gather}
    \begin{cases}
        f_0(x) \to \min \\
        f_i(x) \leq 0, \; i = 1,...,m
    \end{cases}
\end{gather}
Усі функції $f_0, f_1, ..., f_m \in C^1 (\mathbb{R}^n)$ --- неперервно диференційовні на $\mathbb{R}^n$.

\noindent\textbf{Результати роботи.}
Опишемо суть методу. Покладемо $\Phi_N (x) = f(x) + N\cdot F(x)$ для великого $N>0$. $F(x) = \max\left\{0, f_1(x), ..., f_m(x)\right\} \geq 0$
для всіх $x \in \mathbb{R}^n$, причому рівність досягається тоді і тільки тоді, коли $f_i(x) \leq 0, \; i = 1,...,m$.
$N$ є одним з параметрів методу. Нехай $x_0 \in \mathbb{R}^n$ --- довільна початкова точка. На $k$-тому кроці роботи методу спочатку розв'язується
задача квадратичного програмування відносно $p$:
\begin{gather}\label{2}
    \begin{cases}
        \frac{1}{2} \Vert p \Vert^2 + \left(f'_0(x_k), p\right) \to \underset{x}{\min} \\
        f_i(x_k) + \left(f'_i(x_k), p\right) \leq 0, \; i = 1, ..., m
    \end{cases}
\end{gather}
Позначимо
\begin{gather*}
    D = \begin{pmatrix}
        -f'_1 (x_k) \\
        -f'_2 (x_k) \\
        \dots \\
        -f'_m(x_k)
    \end{pmatrix} = 
    \begin{pmatrix}
        -(\mathrm{grad}f_1 (x_k))^T \\
        -(\mathrm{grad}f_2 (x_k))^T \\
        \dots \\
        -(\mathrm{grad}f_m (x_k))^T
    \end{pmatrix},
    f = \begin{pmatrix}
        f_1(x_k) \\
        f_2(x_k) \\
        \dots \\
        f_m(x_k)
    \end{pmatrix},
    c = f'_0(x_k)
\end{gather*}
Тоді задача (\ref{2}) запишеться як
\begin{gather}\label{3}
    \begin{cases}
        \frac{1}{2} \Vert p \Vert^2 + \left(c, p\right) \to \underset{p}{\min} \\
        D p \geq f
    \end{cases}
\end{gather}
Ця задача є лінійним наближенням вихідної задачі в околі точки $x_k$.
Якщо $p_k$ --- розв'язок цієї задачі, то $x_{k+1} = x_k + \alpha_k p_k$, де $\alpha_k$ --- перше число в послідовності $1, \frac{1}{2}, \frac{1}{4}, ...$,
при якому виконується нерівність
\begin{gather}
    \Phi_N (x_k + \alpha_k p_k) \leq \Phi_N(x_k) - \varepsilon \alpha_k \Vert p_k \Vert^2
\end{gather}

Б. М. Пшеничний в книзі <<Метод лінеаризації>> \cite{1} доводить, що точка $x\in\mathbb{R}^n$ є стаціонарною для вихідної задачі тоді і тільки тоді, коли
задача квадратичного програмування
\begin{gather}
    \begin{cases}
        \frac{1}{2} \Vert p \Vert^2 + \left(f'_0(x), p\right) \to \underset{p}{\min} \\
        f_i(x) + \left(f'_i(x), p\right) \leq 0, \; i = 1, ..., m
    \end{cases}
\end{gather}
має розв'язок $p(x)$ і він рівний нулю. Там само доводиться, що за деяких умов (наприклад, ліпшицевість градієнтів функцій, що задають обмеження), метод є збіжним.

Квадратичну задачу (\ref{3}), що виникає на кожному кроці роботи методу, можна замінити так званою дуальною (двоїстою) задачею:
\begin{gather}
    \begin{cases}
        -\frac{1}{2} \Vert u \Vert^2 + \left(f, v\right) \to \underset{u, v}{\max} \\
        D^T v - u = c\\
        v \geq 0
    \end{cases} \Leftrightarrow
    \begin{cases}
        \frac{1}{2} \Vert u\Vert^2 - \left(f, v\right) \to \underset{u, v}{\min} \\
        D^T v - u = c \\
        v \geq 0
    \end{cases}
\end{gather}
Виключивши з задачі $u = D^T v - c$, отримаємо
\begin{gather}
    \begin{cases}
        \frac{1}{2}\left(D D^T v, v\right) - \left(D c + f, v \right) \to \underset{v}{\min} \\
        v \geq 0
    \end{cases}
\end{gather}
Якщо $v^*$ --- її розв'язок, то $p = u^* = D^T v^* - c$ буде розв'язком вихідної квадратичної задачі, як показано в \cite{1} та \cite{2}.

\noindent\textbf{Лістинг.}
%Текст програми було розділено на \texttt{Optimizer.py} з реалізацією
%власне методу проекції градієнту і методу Ньютона з попередньої лабораторної роботи, який використовується
%для розв'язання задачі проекції, та \texttt{lab3.py}, де викликаються необхідні функції та зберігаються результати.

%\noindent\texttt{Optimizer.py}
%\lstinputlisting[language = Python, style = python_code]{../code/Optimizer.py}

%\noindent\texttt{lab4.py}
%\lstinputlisting[language = Python, style = python_code]{../code/lab3.py}

\noindent\textbf{Висновки.}

\begin{thebibliography}{3}
    \bibitem{1} Пшеничный Б.Н. \emph{Метод линеаризации}. --- М.: Наука. Главная редакция физико-математической литературы, 1983. --- 136 с.
    \bibitem{2} Dorn, W., 1960. Duality in quadratic programming. \emph{Quarterly of Applied Mathematics}, 18(2), pp.155-162.
    \bibitem{3} Sha, F., Lin, Y., Saul, L. and Lee, D., 2007. Multiplicative Updates for Nonnegative Quadratic Programming. Neural Computation, 19(8), pp.2004-2031.
\end{thebibliography}
\end{document}