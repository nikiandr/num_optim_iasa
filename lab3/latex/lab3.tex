\documentclass{extreport}

\usepackage[14pt]{extsizes}
\usepackage[T2A]{fontenc}
\usepackage[utf8]{inputenc}
\usepackage[english,ukrainian]{babel}

\usepackage[a4paper, top=10mm, bottom=15mm, left=20mm, right=15mm]{geometry}
\usepackage{amsmath,amsfonts,amssymb,amsthm,mathtools}
\usepackage{listings}
\usepackage{graphicx}
\usepackage{enumitem}
\usepackage{verbatim}
\usepackage{listings}
\usepackage{xcolor}
\usepackage{textgreek}
\usepackage{diagbox}
\usepackage{pgfplots}
\pgfplotsset{compat = 1.16}


\lstdefinestyle{output_txt}{
    basicstyle=\ttfamily\footnotesize,
    breakatwhitespace=false,         
    breaklines=true,                 
    captionpos=b,                    
    keepspaces=true,                                    
    numbersep=5pt,                  
    showspaces=false,                
    showstringspaces=false,
    showtabs=false,                  
    tabsize=2
}
\definecolor{codegreen}{rgb}{0,0.6,0}
\definecolor{codegray}{rgb}{0.5,0.5,0.5}
\definecolor{codepurple}{rgb}{0.58,0,0.82}
\lstdefinestyle{python_code}{ 
    commentstyle=\color{codegreen},
    keywordstyle=\color{magenta},
    numberstyle=\tiny\color{codegray},
    stringstyle=\color{codepurple},
    basicstyle=\ttfamily\footnotesize,
    breakatwhitespace=false,         
    breaklines=true,                 
    captionpos=b,                    
    keepspaces=true,                            
    numbersep=5pt,                  
    showspaces=false,                
    showstringspaces=false,
    showtabs=false,                  
    tabsize=4
}

\setlist[enumerate]{nosep}
\graphicspath{{pics/}}
\DeclareGraphicsExtensions{.png}

\begin{document}
\begin{titlepage}
    \thispagestyle{empty}
    \begin{center}
        \includegraphics[width = \textwidth]{kpi}
        Міністерство освіти і науки України\\
        Національний технічний університет України\\
        <<Київський політехнічний інститут ім. І. Сікорського>>\\
        Інститут прикладного системного аналізу
    \end{center}
    \vspace{40mm}
    \begin{center}
        \textbf{Лабораторна робота} \\
        з курсу <<Методи оптимізації>> \\
        з теми <<Чисельні методи нелінійної умовної оптимізації. 
        Метод проекції градієнта>>
    \end{center}
    \vspace{20mm}
    \begin{flushleft}
        Виконали студенти 3 курсу групи КА-81 \\
        Галганов Олексій \\
        Єрко Андрій \\
        Фордуй Нікіта
    \end{flushleft}
    \begin{flushright}
        Перевірили \\
        Спекторський Ігор Якович \\
        Яковлева Алла Петрівна
    \end{flushright}
    \vspace{30mm}
    \begin{center}
        \textbf{Київ 2021}
    \end{center}
\end{titlepage}

\begin{center}
    \textbf{Варіанти 1 та 9}
\end{center}
\textbf{Завдання.} Скласти програму для умовної мінімізації цільової функції $f$ на множині $X$
одним з методів проекції градієнта. Конкретний тип методу обрати
самостійно, ураховуючи особливості цільової функції.

Цільові функції:

\emph{Варіант 1.} $f(x, y, z) = x^2 + y^2 + z^2$, $X = \{ (x,y,z) \in \mathbb{R}^3 : x + y + z = 1 \}$.

\emph{Варіант 9.} $f(x, y, z) = x + 4y + z$, $X = \{ (x,y,z) \in \mathbb{R}^3 : x^2 + 3y^2 + 2z^2 \leq 1 \}$.

\noindent\textbf{Результати роботи.}

\emph{Варіант 1.}
Для дослідження цільової функції розглянемо її матрицю Гессе
$f''(x,y,z) = \begin{pmatrix}
    2 & 0 & 0 \\
    0 & 2 & 0 \\
    0 & 0 & 2
\end{pmatrix}$. Очевидно, що функція є строго опуклою,
а її матриця Гессе не є погано обумовленою.

Множина $X$ є гіперплощиною в $\mathbb{R}^3$, а отже є замкненою та опуклою.
Явна формула проекції на гіперплощину
$H_{p\beta} = \{x \in \mathbb{R}^n : (p,x)=\beta\}$:
$\pi_X a = a + (\beta - (p,a))\cdot \frac{p}{\|p\|^2}$.
В нашому випадку $\beta = 1$, а $p=(1,1,1)^T$.

Розв'яжемо задачу аналітично:
\begin{gather*}
    \mathcal{L}(x, y, z, \lambda) = x^2 + y^2 + z^2 + \lambda(x+y+z-1) \\
    \begin{cases}
        \frac{\partial \mathcal{L}}{\partial x} = 2x + \lambda = 0 \\
        \frac{\partial \mathcal{L}}{\partial x} = 2y + \lambda = 0 \\
        \frac{\partial \mathcal{L}}{\partial x} = 2z + \lambda = 0 \\
        x + y + z = 1
    \end{cases} \Rightarrow
    \begin{cases}
        x^* = 1/3 \\
        y^* = 1/3 \\
        z^* = 1/3 \\
        \lambda = -2/3
    \end{cases}
\end{gather*}

\emph{Варіант 9.} В цьому випадку мінімізується неперервна функція на компакті, тому розв'язок у задачі існує. Знайти явну функцію для обчислення проекції
на еліпсоїд складно або навіть неможливо.
Зауважимо, що цільова функція є гармонічною у заданій множині 
(тобто, $\Delta f = \frac{\partial^2 f}{\partial x^2} + \frac{\partial^2 f}{\partial y^2} + \frac{\partial^2 f}{\partial z^2} = 0$),
тому досягати екстремуму може лише на границі множини $\partial X = \{ (x,y,z) \in \mathbb{R}^3 : x^2 + 3y^2 + 2z^2 = 1 \}$. 
Отже, будемо шукати проекції чисельно, застосувавши \emph{метод штрафних функцій}, який полягає у тому, що задача умовної оптимізації
$$ 
\begin{cases}
    f(x) \to \min \\
    g_1(x) = 0, ..., g_k(x) = 0
\end{cases}
$$
замінюється задачею безумовної оптимізації
$$
\widehat{f}(x) = f(x) + \alpha \cdot \left(g_1^2(x) + ... + g_k^2(x)\right) \to \min, \; \alpha \gg 0
$$
Суть цього методу у тому, що другий доданок при великих значеннях $\alpha$ наближує індикатор множини, що задається обмеженнями умовної задачі.

Розв'яжемо задачу аналітично:
\begin{gather*}
    \mathcal{L}(x, y, z, \lambda) = x + 4y + z + \lambda(x^2 + 3y^2 + 2z^2 - 1) \\
    \begin{cases}
        \frac{\partial \mathcal{L}}{\partial x} = 1 + 2\lambda x = 0 \\
        \frac{\partial \mathcal{L}}{\partial x} = 4 + 6\lambda y = 0 \\
        \frac{\partial \mathcal{L}}{\partial x} = 1 + 4\lambda z = 0 \\
        x^2 + 3y^2 + 2z^2 = 1
    \end{cases} \Rightarrow
    \begin{cases}
        x = -1/{2\lambda} \\
        y = -2/{3\lambda} \\
        z = -1/{4\lambda} \\
        x^2 + 3y^2 + 2z^2 = 1
    \end{cases} \Rightarrow \\ \Rightarrow
    \begin{cases}
        x = -1/{2\lambda} \\
        y = -2/{3\lambda} \\
        z = -1/{4\lambda} \\
        \lambda = \pm \sqrt{41/24}
    \end{cases} \Rightarrow
    \begin{cases}
        x = \mp \sqrt{6/41} \\
        y = \mp 4\sqrt{2/123} \\
        z = \mp \sqrt{3/82} \\
        \lambda = \pm \sqrt{41/24}
    \end{cases}
\end{gather*}
Отже, розв'язок задачі --- $\left(-\sqrt{\frac{6}{41}}, -4\sqrt{\frac{2}{123}}, -\sqrt{\frac{3}{82}}\right) \approx \left(-0.383, -0.51, -0.191\right)$.

Для пришвидшення збіжності методу проекції, для реалізації було обрано метод дроблення кроку.
Критерій зупинки --- $\left| f(x_{k+1}) - f(x_k)\right| < \varepsilon = 10^{-5}$.

\begin{center}
    \begin{tabular}{|c|c|c|}
        \hline
        {} & \multicolumn{2}{c|}{кількість ітерацій}\\
        \hline
        $x_0$ & {варіант 1} & {варіант 9}\\
        \hline
        $\mathrm{argmin}$ & 2 & 2 \\
        \hline
        $(0, 0, 0)$ & 2 & 31 \\
        \hline
        $(1, 1, 1)$ & 2 & 38 \\
        \hline
        $(10, 10, 10)$ & 2 & 38 \\
        \hline
        $(100, 100, 100)$ & 2 & 39 \\
        \hline
        $(1000, 1000, 1000)$ & 2 & 40 \\
        \hline
    \end{tabular}
\end{center}

Приклади траєкторій спуску методу:
\begin{center}
    \begin{tabular}{c c}
        \includegraphics[scale = 0.7]{res_var1_(1).png} &
        \includegraphics[scale = 0.7]{res_var9_(1).png} 
    \end{tabular}
\end{center}

\noindent\textbf{Лістинг.}
Текст програми було розділено на \texttt{Optimizer.py} з реалізацією
власне методу проекції градієнту та \texttt{lab3.py}, де він викликається та зберігаються результати.

\noindent\texttt{Optimizer.py}
\lstinputlisting[language = Python, style = python_code]{../code/Optimizer.py}

\noindent\texttt{lab3.py}
\lstinputlisting[language = Python, style = python_code]{../code/lab3.py}

\noindent\textbf{Висновки.} При виконанні даної роботи ми дослідили
застосування методу проекції градієнта до мінімізації квадратичної функції 
з обмеженням на гіперплощині та лінійної функції з обмеженням на еліпсоїді. Для збільшення швидкості збіжності було реалізовано метод дроблення кроку,
який на кожній ітерації обирає крок $\alpha_k$ найбільшим серед тих, які забезпечують рух
у напрямку мінімуму. Як ми побачили, метод збігається навіть тоді, коли початкова точка не належить множині $X$.
\end{document}