\documentclass{extreport}

\usepackage[14pt]{extsizes}
\usepackage[T2A]{fontenc}
\usepackage[utf8]{inputenc}
\usepackage[english,ukrainian]{babel}

\usepackage[a4paper, top=10mm, bottom=15mm, left=20mm, right=15mm]{geometry}
\usepackage{amsmath,amsfonts,amssymb,amsthm,mathtools}
\usepackage{listings}
\usepackage{graphicx}
\usepackage{enumitem}
\usepackage{verbatim}
\usepackage{listings}
\usepackage{xcolor}
\usepackage{textgreek}
\usepackage{diagbox}
\usepackage{pgfplots}
\pgfplotsset{compat = 1.16}


\lstdefinestyle{output_txt}{
    basicstyle=\ttfamily\footnotesize,
    breakatwhitespace=false,         
    breaklines=true,                 
    captionpos=b,                    
    keepspaces=true,                                    
    numbersep=5pt,                  
    showspaces=false,                
    showstringspaces=false,
    showtabs=false,                  
    tabsize=2
}
\definecolor{codegreen}{rgb}{0,0.6,0}
\definecolor{codegray}{rgb}{0.5,0.5,0.5}
\definecolor{codepurple}{rgb}{0.58,0,0.82}
\lstdefinestyle{python_code}{ 
    commentstyle=\color{codegreen},
    keywordstyle=\color{magenta},
    numberstyle=\tiny\color{codegray},
    stringstyle=\color{codepurple},
    basicstyle=\ttfamily\footnotesize,
    breakatwhitespace=false,         
    breaklines=true,                 
    captionpos=b,                    
    keepspaces=true,                            
    numbersep=5pt,                  
    showspaces=false,                
    showstringspaces=false,
    showtabs=false,                  
    tabsize=4
}

\setlist[enumerate]{nosep}
\graphicspath{{pics/}}
\DeclareGraphicsExtensions{.png}

\begin{document}
\begin{titlepage}
    \thispagestyle{empty}
    \begin{center}
        \includegraphics[width = \textwidth]{kpi}
        Міністерство освіти і науки України\\
        Національний технічний університет України\\
        <<Київський політехнічний інститут ім. І. Сікорського>>\\
        Інститут прикладного системного аналізу
    \end{center}
    \vspace{40mm}
    \begin{center}
        \textbf{Лабораторна робота} \\
        з курсу <<Методи оптимізації>> \\
        з теми <<Чисельні методи нелінійної умовної оптимізації. 
        Метод проекції градієнта>>
    \end{center}
    \vspace{20mm}
    \begin{flushleft}
        Виконали студенти 3 курсу групи КА-81 \\
        Галганов Олексій \\
        Єрко Андрій \\
        Фордуй Нікіта
    \end{flushleft}
    \begin{flushright}
        Перевірили \\
        Спекторський Ігор Якович \\
        Яковлева Алла Петрівна
    \end{flushright}
    \vspace{30mm}
    \begin{center}
        \textbf{Київ 2021}
    \end{center}
\end{titlepage}

\begin{center}
    \textbf{Варіант 1}
\end{center}
\textbf{Завдання.} Скласти програму для умовної мінімізації цільової функції
одним з методів проекції градієнта. Конкретний тип методу обрати
самостійно, ураховуючи особливості цільової функції.

\emph{Цільова функція:}
$f(x,y) = x^2 + y^2 + z^2$

\emph{Обмеження:}
$X = \{ (x,y,z) \in \mathbb{R}^3 \mid x + y + z = 1 \}$

\noindent\textbf{Результати роботи.}
Для дослідження цільової функції розглянемо її матрицю Гессе
$f''(x,y,z) = \begin{pmatrix}
    2 & 0 & 0 \\
    0 & 2 & 0 \\
    0 & 0 & 2
\end{pmatrix}$. Очевидно, що функція є строго опуклою,
а її матриця Гессе не є погано обумовленою.

Множина $X$ є гіперплощиною в $\mathbb{R}^3$, а отже є замкненою та опуклою.
Явна формула проекції на гіперплощину
$H_{p\beta} = \{x \in \mathbb{R}^n \mid (p,x)=\beta\}$:
$\pi_X\alpha = \alpha + (\beta - (p,a))\cdot \frac{p}{\|p\|^2}$

В нашому випадку $\beta = 1$, а $p=(1,1,1)^T$

Для пришвидшення збіжності, для реалізації було обрано метод дроблення кроку,
який має допомогти зменшити зигзагоподібний характер шляху до точки мінімуму.
Обраний критерій зупинки --- $\left| f(x_{k+1}) - f(x_k)\right| < \varepsilon = 10^{-5}$.
Було отримано збіжність до точки мінімуму за $54$ ітерацій.

\lstinputlisting[style = output_txt]{../output.txt}

\noindent\textbf{Лістинг.}
Текст програми було розділено на \texttt{Optimizer.py} з реалізацією
власне методу проекції градієнту та \texttt{lab3.py}, де він викликається та зберігаються результати.

\noindent\texttt{Optimizer.py}
\lstinputlisting[language = Python, style = python_code]{../code/Optimizer.py}

\noindent\texttt{lab3.py}
\lstinputlisting[language = Python, style = python_code]{../code/lab3.py}

\noindent\textbf{Висновки.} При виконанні даної роботи ми дослідили
застосування методу проекції градієнта до мінімізації заданої квадратичної функції 
з обмеженням на гіперплощині. Для збільшення швидкості збіжності було реалізовано метод дроблення кроку,
який на кожній ітерації обирає крок $\alpha_k$ найбільшим серед тих, які забезпечують рух
у напрямку мінімуму.
\end{document}